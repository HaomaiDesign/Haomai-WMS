\documentclass[12pt,a4paper]{article}
\usepackage[spanish]{babel}
\usepackage[utf8]{inputenc}
\usepackage{enumitem}
\usepackage{geometry}
\geometry{margin=2.5cm}
\title{Haomai -- Documentación General del Sistema}
\author{}
\date{\today}

\begin{document}
\maketitle

\section*{Introducción}
Haomai es un sistema integral para la gestión de depósitos y control de stock, diseñado para optimizar el flujo de mercadería, reducir errores y brindar información precisa en tiempo real. Su enfoque está orientado a empresas que necesitan controlar ingresos y egresos de productos, gestionar inventarios, planificar entregas y tomar decisiones basadas en reportes confiables.

\section*{Objetivo del Sistema}
El objetivo principal de Haomai es centralizar y simplificar la administración de depósitos, permitiendo a los usuarios registrar, consultar y analizar todos los movimientos de mercadería, desde la recepción hasta la entrega final, con trazabilidad total y herramientas de análisis para distintos perfiles de usuario.

\section*{Módulos Principales}

\subsection*{1. Ingreso y Egreso de Mercadería}
Permite registrar de manera ágil y precisa todas las entradas y salidas de productos en el depósito. Incluye:
\begin{itemize}[leftmargin=*, itemsep=0.5em]
  \item Escaneo de códigos de barra para agilizar la carga de productos.
  \item Selección de depósito y lote para cada movimiento.
  \item Asignación y control de fechas de vencimiento.
  \item Selección de lote específico al egresar productos (rotación eficiente).
\end{itemize}
Este módulo es fundamental para mantener actualizado el stock y garantizar la trazabilidad de cada artículo.

\subsection*{2. Stock e Inventario}
Ofrece una visión centralizada y en tiempo real del inventario disponible en cada depósito y ubicación. Permite:
\begin{itemize}[leftmargin=*, itemsep=0.5em]
  \item Consultar stock actualizado por producto, lote y ubicación.
  \item Visualizar el historial de movimientos de cada artículo.
  \item Gestionar múltiples depósitos y ubicaciones internas.
\end{itemize}
Facilita la preparación de pedidos, la reposición y el control de inventario físico.

\subsection*{3. Logística y Entregas}
Gestiona la planificación y el seguimiento de entregas de mercadería:
\begin{itemize}[leftmargin=*, itemsep=0.5em]
  \item Organización de pedidos para despacho.
  \item Asignación de entregas y seguimiento del estado.
  \item Registro de entregas realizadas y pendientes.
\end{itemize}
Este módulo conecta el depósito con el cliente final o sucursal de destino, asegurando la trazabilidad hasta la entrega.

\subsection*{4. Reportes, Relatorios y Alertas}
Brinda herramientas de análisis y control para distintos perfiles:
\begin{itemize}[leftmargin=*, itemsep=0.5em]
  \item Reportes de stock por categoría y movimientos (para gerencia).
  \item Reportes de disponibilidad y ubicación (para empleados).
  \item Listados de productos próximos a vencer, caducados y alertas de stock bajo.
\end{itemize}
Permite tomar decisiones informadas y anticiparse a problemas de inventario.

\subsection*{5. Gestión de Clientes}
Administra la base de datos de clientes y destinos de entrega:
\begin{itemize}[leftmargin=*, itemsep=0.5em]
  \item Registro y consulta de clientes y sucursales.
  \item Asociación de pedidos y entregas a cada cliente.
\end{itemize}
Facilita la trazabilidad y el control de las relaciones comerciales.

\subsection*{6. Gestión de roles y permisos}
El sistema permite definir distintos perfiles de usuario (por ejemplo: gerencia, empleados, operadores) y asignar permisos específicos a cada uno. Esto posibilita:
\begin{itemize}[leftmargin=*, itemsep=0.5em]
  \item Controlar el acceso a cada módulo o funcionalidad según el rol.
  \item Garantizar la seguridad y confidencialidad de la información.
  \item Adaptar la experiencia de uso a las responsabilidades de cada usuario.
\end{itemize}

\section*{Resumen}
Haomai integra todos los procesos clave de un depósito en una sola plataforma, brindando control, eficiencia y visibilidad total sobre la operación logística y de inventario.

\end{document} 