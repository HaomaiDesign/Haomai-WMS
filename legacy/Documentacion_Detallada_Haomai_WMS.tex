\documentclass[12pt,a4paper]{article}
\usepackage[spanish]{babel}
\usepackage[utf8]{inputenc}
\usepackage{enumitem}
\usepackage{geometry}
\geometry{margin=2.5cm}
\title{Documentación detallada de funcionamiento interno de Haomai WMS}
\author{}
\date{\today}

\begin{document}
\maketitle

\section*{1. Stack tecnológico y versiones}
\begin{itemize}
  \item \textbf{Lenguaje principal:} PHP 7.x/8.x
  \item \textbf{Frontend:} HTML5, CSS3, JavaScript (jQuery)
  \item \textbf{Frameworks/librerías:} Bootstrap, Toastr, FontAwesome, Ionicons
  \item \textbf{Base de datos:} MySQL/MariaDB
  \item \textbf{Contenedores:} Docker, Docker Compose
  \item \textbf{Dependencias externas:}
    \begin{itemize}
      \item aws/aws-sdk-php (v2)
      \item picqer/php-barcode-generator (\^2.4)
    \end{itemize}
\end{itemize}

\section*{2. Estructura de carpetas y finalidad}
\begin{itemize}[leftmargin=*, itemsep=0.5em]
  \item \textbf{assets/} -- Recursos estáticos: CSS, JS, imágenes, fuentes.
  \item \textbf{company/} -- Pantallas y lógica de gestión de empresas.
  \item \textbf{logistic/} -- Módulo de logística: rutas, entregas, seguimiento.
  \item \textbf{market/} -- Módulo de marketplace: productos, pedidos, órdenes, reportes.
  \item \textbf{reports/} -- Generación y exportación de reportes (Excel, PDF).
  \item \textbf{stock/} -- Gestión de inventario: productos, movimientos, tareas, depósitos.
  \item \textbf{system/} -- Archivos de sistema: sesión, base de datos, utilidades.
  \item \textbf{users/} -- Gestión de usuarios, roles y permisos.
  \item \textbf{vendor/} -- Dependencias instaladas por Composer.
  \item \textbf{webservice/} -- Endpoints y lógica de API interna.
\end{itemize}

\section*{3. Dockerización y despliegue}
\subsection*{3.1. Archivos clave}
\begin{itemize}
  \item \textbf{docker-compose.yml:} Define los servicios (web, db) y sus redes/volúmenes.
  \item \textbf{.env:} Variables de entorno para conexión a base de datos y configuración.
\end{itemize}
\subsection*{3.2. Pasos para levantar el sistema}
\begin{enumerate}
  \item Clonar el repositorio y ubicarse en la raíz del proyecto.
  \item Configurar el archivo .env con los datos de la base de datos.
  \item Ejecutar: \texttt{docker-compose up -d}
  \item Acceder a la aplicación desde el navegador en la URL indicada (por defecto, http://localhost:8080)
\end{enumerate}
\subsection*{3.3. Consideraciones}
\begin{itemize}
  \item El contenedor web utiliza PHP y Apache.
  \item El contenedor db utiliza MySQL/MariaDB y persiste datos en un volumen.
  \item Para desarrollo, los cambios en archivos fuente se reflejan automáticamente.
\end{itemize}

\section*{4. Módulos y pantallas principales (según menú)}
\subsection*{4.1. Pedidos}
\begin{itemize}
  \item \textbf{Pedidos:} Visualización y gestión de órdenes de ingreso/egreso.
\end{itemize}
\subsection*{4.2. Ingreso de Mercadería}
\begin{itemize}
  \item Permite registrar la entrada de productos, escanear códigos de barra, asignar depósito y lote, y definir fecha de vencimiento.
\end{itemize}
\subsection*{4.3. Egreso de Mercadería}
\begin{itemize}
  \item Permite registrar la salida de productos, seleccionar lote a egresar (si hay varios), y asociar el movimiento a un pedido o cliente.
\end{itemize}
\subsection*{4.4. Logística}
\begin{itemize}
  \item \textbf{Pedidos:} Listado de pedidos listos para entrega.
  \item \textbf{Entrega:} Gestión y seguimiento de entregas a clientes o sucursales.
\end{itemize}
\subsection*{4.5. Depósito}
\begin{itemize}
  \item \textbf{Lista de productos:} Catálogo de productos en stock.
  \item \textbf{Disponibilidad:} Consulta de stock por producto y ubicación.
  \item \textbf{Registros:} Historial de movimientos de inventario.
  \item \textbf{Ubicación:} Gestión de ubicaciones físicas dentro del depósito.
\end{itemize}
\subsection*{4.6. Clientes}
\begin{itemize}
  \item \textbf{Listado de Clientes:} ABM de clientes y consulta de historial de pedidos/entregas.
\end{itemize}
\subsection*{4.7. Reporte}
\begin{itemize}
  \item \textbf{Por categoría:} Reporte de stock agrupado por familia de productos.
  \item \textbf{Fecha de Vencimiento:} Listado de productos próximos a vencer.
  \item \textbf{Caducado:} Productos vencidos en stock.
  \item \textbf{Alerta de inventario:} Productos con stock bajo.
  \item \textbf{Reporte de movimiento de stock:} Historial detallado de ingresos, egresos y ajustes.
\end{itemize}

\section*{5. Resumen de utilidad de cada sección}
\begin{itemize}[leftmargin=*, itemsep=0.5em]
  \item \textbf{Pedidos:} Centraliza la gestión de órdenes y su estado.
  \item \textbf{Ingreso/Egreso:} Mantiene actualizado el stock y asegura trazabilidad.
  \item \textbf{Logística:} Permite planificar y controlar entregas.
  \item \textbf{Depósito:} Brinda visibilidad y control sobre el inventario físico.
  \item \textbf{Clientes:} Facilita la gestión comercial y la trazabilidad de entregas.
  \item \textbf{Reportes:} Provee información clave para la toma de decisiones y auditoría.
\end{itemize}

\end{document} 